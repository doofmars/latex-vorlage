\chapter{Einleitung}

\section{Strukturen}

\subsection{asdasd}
asdfasdf

\subsubsection{Subsubsection}
asdf

\paragraph{Paragraph}
asdf

\section{Beispielabbildungen}

\lipsum[10]

\begin{wrapfigure}{rt}{8cm}
\caption{Bildüberschrift}
\centering
\includegraphics[width=0.3\textwidth]{content/pictures/hfu}
Quelle: \cite{s11wasml}
\label{pic:bild2}
\end{wrapfigure}

\lipsum[10]

\begin{figure}
\caption{Bildüberschrift}
\includegraphics[width=1\textwidth]{content/pictures/hfu}
Quelle: \cite{s11wasml}
\label{pic:bild1}
\end{figure}

\section{Beispieltabelle}

\begin{table}
\caption{Tabellenüberschrift}
\center
\footnotesize
\begin{tabular}{lll}
\toprule
Head1 & Head2 & Head3 \\
\midrule
Val1 & Val2 & Val3 \\
Val4 & Val5 & Val6 \\
\bottomrule
\end{tabular}
\end{table}

\section{Listings}

\begin{lstlisting}[language=java, caption=Hallo Welt in Java]
public class HalloWelt 
{
	public static void main(String[] args) 
	{
		System.out.println("Hallo Welt!");
	}
}
\end{lstlisting}

\section{Abkürzungen}

Im Abkürzungsverzeichnis stehende Abkürzungen können in Langform (\ac{HFU}) oder in Kurzform (\acs{HFU}) angegeben werden.

\section{Abgesetztes wörtliches Zitat}

\lipsum[10]

\begin{quote}
\textit{\enquote{Eingerücktes wörtliches Zitat.}}\cite[S. 14ff]{s11wasml}
\end{quote}

\lipsum[10]

\section{Theoreme}

\lipsum[2]

\begin{beispiel}
Beispieltext\dots
\end{beispiel}
 
\lipsum[2]

\begin{these}
These\dots
\end{these}
 
\lipsum[2]
 
\begin{definition}
Unter dem Begriff \dots verstehen wir \dots
\end{definition}

\lipsum[2]